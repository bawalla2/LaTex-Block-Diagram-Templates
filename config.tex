% ************************************************************************
% ************************************************************************
% ************************************************************************
%
% CONFIG
%
% ************************************************************************
% ************************************************************************
% ************************************************************************

\documentclass{article}

% ************************************************************************
%
% STANDARD PACKAGES
%
% ************************************************************************

% Simple arithmetic in latex commands
\usepackage{calc}

% Logical operations
\usepackage{ifthen}


% ************************************************************************
%
% GENERAL FORMATTING PACKAGES
%
% ************************************************************************

% Additional listings, enumaration options
\usepackage{listings}		
\usepackage{enumerate}	
\usepackage[shortlabels]{enumitem}

% Hyperlinked references
\usepackage[%dviwindo,%
breaklinks=true,%
colorlinks=true,%
hyperindex=true,%
plainpages=false,%
linkcolor=black,%
citecolor=black]{hyperref}

% Extended verbatim environments
\usepackage{fancyvrb}
%\usepackage[fancyvrb]{listings}

% Dropped capital letter at beginning of paragraph
\usepackage{lettrine}

% ***********************
%
% HIGHLIGHTING
%

% Colored text, highlighted text
\usepackage{soul}

% For highlighting text, tables
\usepackage[table]{xcolor}

% Large color box environments
\usepackage[breakable]{tcolorbox}


% ***********************
%
% PAGE LAYOUT
%

\voffset=0.5in%
\topmargin=-1.0in%
\textheight=9.0in %

\hoffset=0.0in %
\textwidth=6.5in %
\oddsidemargin=0.0in %
\evensidemargin=0.0in %

% Initialize page counter
\setcounter{page}{1}

% ***********************
%
% MISC
%
% From previous versions of the lab assignments
%

\usepackage{curves}
\usepackage{fancyhdr}
\usepackage{eepic,eepicemu}
\usepackage{makeidx}

\lstloadlanguages{Matlab,C++}

\newboolean{inappendix}
\setboolean{inappendix}{false}

% ************************************************************************
%
% MATH/EQUATION PACKAGES AND FORMATTING
%
% ************************************************************************

% Basic math symbols
\usepackage{gensymb}
\usepackage{amssymb}

% Math tools
\usepackage{mathtools}
\usepackage{mathrsfs}
\usepackage{amsmath, amsfonts}

% Math definitions
\newcommand{\defas} {\buildrel \rm def \over =}
\newcommand{\magn}[1]{\left\vert #1\right\vert}
\newcommand{\norm}[1]{\left\Vert #1\right\Vert}
\newcommand{\Linf}{{\cal L}^\infty}
\newcommand{\Ltwo}{{\cal L}^2}
\newcommand{\Hinf}{{\cal H}^\infty}
\newcommand{\RHinf}{{\cal R}\Hinf}
\newcommand{\Htwo}{{\cal H}^2}
\newcommand{\Lone}{{\cal L}^1}
\newcommand{\Hone}{{\cal H}^1}
\newcommand{\wstar} {\buildrel {\rm weak}^*  \over \longrightarrow}
\newcommand{\compact} {\buildrel {\rm compact}  \over \longrightarrow}

% The following package (accents) and command (ubar) is added to get ``bar'' underneath some symbols/text in Math environment (E.g., for indicating minimum singular value \ubar{\sigma})
\usepackage{accents} 
\newcommand{\ubar}[1]{\underaccent{\bar}{#1}}



% ************************************************************************
%
% FIGURE/TABLE FORMATTING PACKAGES
%
% ************************************************************************


\usepackage{graphicx}

% Figure controls
\usepackage{caption}
\usepackage{subcaption}

% Figure placement
\usepackage{float}
\usepackage{placeins} 

% Wrap figures/tables in text (i.e., Di Vinci style)
\usepackage{wrapfig}

% Suppress sub-figure labeling: (a), (b), etc.
\captionsetup[subfigure]{labelformat=empty}

% For tables
\usepackage{hhline} 		% Custom horizontal lines in tables
\usepackage{multirow}		% Multirow feature in tables
\usepackage{chngpage}
\usepackage{lscape} 		% Turn page contents to landscape
%\usepackage{slashbox} 		% For boxes with a slash and 2 subcells (e.g., for axis labels)

% For footnotes
\usepackage{tablefootnote} 	% For footnotes in tables. Use \tablefootnote{} command

% For algorithms
\usepackage{algpseudocode}

% For background highlight in figures
% See:   https://tex.stackexchange.com/questions/510775/how-to-change-the-background-color-of-a-figure-in-latex
%
%\begin{figure}[h]
%	\begin{center}
%	\begin{mdframed}[backgroundcolor=blue!50,linecolor=blue!50]     % ***********
%		\begin{tabular}{l}
%			\includegraphics[height=\figsize]{example-image-a4-landscape.pdf}	
%		\end{tabular}
%	\end{mdframed}                                                  % ***********
%	\end{center}
%	\caption{ZZZ.}
%	\label{fig:}	
%\end{figure}
%
\usepackage{mdframed}

% For increasing padding above/below elements in a table
% https://tex.stackexchange.com/questions/126539/padding-at-the-top-of-a-table-cell-in-latex
\newcommand\Tstrut{\rule{0pt}{2.6ex}}       % "top" strut
\newcommand\Bstrut{\rule[-0.9ex]{0pt}{0pt}} % "bottom" strut
\newcommand{\TBstrut}{\Tstrut\Bstrut} % top&bottom struts
% \begin{tabular}{|l|l|l|}
%  \hline
%  A  & B  & C \TBstrut\\ % top and bottom struts
%  \hline
%  A1 & B1 & C1 \Tstrut\\ % top strut only
%  A2 & B2 & C2 \Bstrut\\ % bottom strut only
%  \hline
% \end{tabular}


% ****** Definition for EPS Inclusion *****************

\newcommand{\seteps}[4]{\hspace*{#1}\relax{
  \includegraphics*[width=#2, height=#3]{#4}\par}
}

\newcommand{\centereps}[3]{\relax{\par\centering
  \includegraphics*[width=#1, height=#2]{#3}\par}
}

% *****************************************************

% ****** Definition for BMP Inclusion *****************

\newdimen\dwidth
\newdimen\dheight

\def\showbmp#1#2#3{%
\dwidth=#2\dheight=#3
\edef\width{\number\dwidth} \edef\height{\number\dheight}%
\special{insertimage: #1 \width \space \height}}

\def\setbmp#1#2#3#4{\vskip#3\relax\noindent\hskip#1\relax
 \showbmp{#4}{#2}{#3}\hfil}

\def\centerbmp#1#2#3{\vskip#2\relax\centerline{\hbox%
to#1{\showbmp{#3}{#1}{#2}\hfil}}}
% *****************************************************

% ****** Definition for WMF Inclusion *****************

\newcommand{\setwmf}[4]{\hspace*{#1}\relax{
  \includegraphics*[width=#2, height=#3]{#4}\par}
}

\newcommand{\centerwmf}[3]{\relax{\par\centering
  \includegraphics*[width=#1, height=#2]{#3}\par}
}

% ************************************************************************
%
% TIKZ
%
% ************************************************************************

\include{\relpathconfig config_tikz}




% ************************************************************************
%
% CUSTOM COMMANDS
%
% ************************************************************************

% UGM, DGM, PM, DM
\DeclareMathOperator{\DGM}{\downarrow \!\! \mathit{GM}}
\DeclareMathOperator{\UGM}{\uparrow \!\! \mathit{GM}}
\DeclareMathOperator{\PM}{\mathit{PM}}
\DeclareMathOperator{\DM}{\mathit{DM}}

