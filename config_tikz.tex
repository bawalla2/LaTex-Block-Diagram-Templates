% ************************************************************************
% ************************************************************************
% ************************************************************************
%
% CONFIG -- TIKZ
%
% ************************************************************************
% ************************************************************************
% ************************************************************************

% ************************************************************************
% ************************************************************************
%
% GENERAL
%
% ************************************************************************
% ************************************************************************

\usepackage{tikz}				% For block diagrams
\usepackage{xfp}				% For floating point calculations

%\usetikzlibrary{spath3}

% Extra arrow heads, snakes, backgrounds
% https://www.bu.edu/math/files/2013/08/tikzpgfmanual.pdf 		pp 35
\usetikzlibrary{arrows,decorations,backgrounds,arrows.meta}
% For calculations
\usetikzlibrary{calc}
% To prevent bending issue when plotting arrows midway through an arc
% https://tex.stackexchange.com/questions/345749/tikz-arc-with-latex-arrow-tip
\usetikzlibrary{bending}
% For calculating angles
% https://tex.stackexchange.com/questions/253200/drawing-a-circular-arc-between-two-rays
\usetikzlibrary{angles,positioning,intersections,quotes,decorations.markings}
% For matrices
% https://tex.stackexchange.com/questions/291714/colored-box-around-a-column-in-matrix
\usetikzlibrary{matrix}
% For "picture" environment commands (e.g., \circle)
% https://www.overleaf.com/learn/latex/Picture_environment
\usepackage[pdftex]{pict2e}

% ************************************************************************
% ************************************************************************
%
% TIKZ -- DIMENSIONS, CONSTANTS, ETC.
%
% ************************************************************************
% ************************************************************************

% ***********************
%
% SIZING
%

\newdimen\Linewidth \Linewidth = 0.2mm
\newdimen\Linewidthbf \Linewidthbf = 0.6mm
\newdimen\Linewidthtick \Linewidthtick = 0.3mm
\newdimen\Linewidthpz \Linewidthpz = 0.4mm
\newdimen\Shiftarrowbf \Shiftarrowbf = 1.5mm
\newdimen\Shiftlabely \Shiftlabely = -5mm
\newdimen\Shiftlabelx \Shiftlabelx = 8mm
\newdimen\Radcirc \Radcirc = 3cm
\newcommand{\Arrowbf}{{latex[width=2mm]}}

% ***********************
%
% SIZING -- BLOCK DIAGRAMS
%

\newcommand\blockdiagramdimendefault{%
\xdef\bdcirclediameter{1}
\xdef\bdarrowlength{4}
\xdef\bdhorizlabeloffsetx{1.5}
\xdef\bdhorizlabeloffsety{0.5}
\xdef\bdvertlabeloffsetx{\bdhorizlabeloffsety}
\xdef\bdvertlabeloffsety{-\bdhorizlabeloffsetx}
\xdef\bdboxwidth{5}
\xdef\bdboxheight{4}
\xdef\bdboxlabeloffsety{-1}
\xdef\feedbackpathheight{6}
\xdef\bdarrowlengthup{2.5}
}

% ***********************
%
% DRAW COMMANDS
%

\def\crosspath{+(-1.4ex,-1.4ex) -- +(1.4ex,1.4ex) +(1.4ex,-1.4ex) -- +(-1.4ex,1.4ex)}
\def\vtick{+(0,-1.4ex) -- +(0,1.4ex)}
\def\htick{+(-1.4ex,0) -- +(1.4ex,0)}
\tikzstyle{zero}=[circle,draw=black,fill=white,line width = \Linewidthpz,inner sep=0pt, minimum size=1.4ex]
\tikzstyle{point}=[circle,draw=black,fill=black,line width = \Linewidthpz,inner sep=0pt, minimum size=0.5ex]



% ************************************************************************
%
% TIKZ -- FUNCTIONALITY COMMANDS
%
% ************************************************************************

% PGF Layers
\pgfdeclarelayer{background}
\pgfdeclarelayer{foreground}
\pgfsetlayers{background,main,foreground}

% ***********************
%
% COORDINATE CALCULATIONS
%

% Current coordinate
% https://tex.stackexchange.com/questions/132924/using-current-path-position-in-coordinate-calculation
\makeatletter
\newcommand\currentcoordinate{\the\tikz@lastxsaved,\the\tikz@lastysaved}
\makeatother
\makeatletter
\newcommand\currentcoordinatex{\the\tikz@lastxsaved}
\makeatother
\makeatletter
\newcommand\currentcoordinatey{\the\tikz@lastysaved}
\makeatother

% Extract x and y values from coordinate
% https://tex.stackexchange.com/questions/56353/extract-x-y-coordinate-of-an-arbitrary-point-on-curve-in-tikz
% \gettikzxy{(<coord name>)}{\<storage name x>}{\<storage name y>}
% E.g.,
% \gettikzxy{(A)}{\ax}{\ay}
% \gettikzxy{(B)}{\bx}{\by}
\makeatletter
\newcommand{\gettikzxy}[3]{%
  \tikz@scan@one@point\pgfutil@firstofone#1\relax
  \edef#2{\the\pgf@x}%
  \edef#3{\the\pgf@y}%
}
\makeatother
%
% (x, y) in cm
%
\newcommand{\gettikzxycm}[3]{%
	\gettikzxy{#1}{#2}{#3}%
	\xdef#2{\fpeval{round(#2*1pt/1cm,6)}}%
	\xdef#3{\fpeval{round(#3*1pt/1cm,6)}}%
}
%
% Return x, y in cm
\newcommand\currentcoordinatexcm{\fpeval{\currentcoordinatex*1pt/1cm}}
\newcommand\currentcoordinateycm{\fpeval{\currentcoordinatey*1pt/1cm}}

% Use macro to set style attributes of a \draw command
% https://tex.stackexchange.com/questions/456259/using-a-macro-with-tikz-style-attributes-in-a-draw-command
% Example:
% \def\styleattributes{fill=blue,opacity=0.8}
% \begin{tikzpicture}
%   \draw[apply style/.expand once=\styleattributes] (0,0) rectangle (1,1);
% \end{tikzpicture}
\tikzset{apply style/.code={\tikzset{#1}}}



% ***********************
%
% MACRO: CHECK IF PGFKEY EMPTY
%
% \ifpgfkeyempty{<key>}{<true case>}{<false case>}
%
% https://tex.stackexchange.com/questions/523512/detecting-empty-pgfkeys
%  

\makeatletter
\def\ifempty#1{%
    \xdef\temp{#1}%
    \ifx\temp\empty
       \expandafter\@firstoftwo
    \else
       \expandafter\@secondoftwo
    \fi
}

\def\ifpgfkeyempty#1{%
    \pgfkeysgetvalue{#1}{\temp}%
    \ifx\temp\empty
       \expandafter\@firstoftwo
    \else
       \ifx\temp\relax
           \expandafter\expandafter\expandafter\@firstoftwo
       \else
           \expandafter\expandafter\expandafter\@secondoftwo
       \fi
    \fi
}
\makeatother

